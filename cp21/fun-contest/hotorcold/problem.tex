\input{.template/template.tex}

\begin{document}

\makeheader{end of contest}

Vitalik is a fan of crypto currency and security.
That's why he uses a hardware wallet to protect his coins/tokens.
However, he somehow lost the wallet and must recover it.
He knows that it must be in an $1024 x 1024$ area around him.
Luckily, he has a transmitter on him that can be used to localize the wallet.

The transmitter has one button, a display and a limited battery capacity.
On the first push of the button, the transmitter initializes itself with a location.
On the succeeding pushes, the transmitter answers with $hot$ if it came nearer to the wallet,
otherwise it answers with  $cold$.  The battery only lasts for $50$ pushes of the button.

Vitalik contacted you to write an algorithm that is able to find the wallet before the battery runs out.

\paragraph*{Interaction}

This is an interactive problem. This means that your program will not first
read some input and then write some output, but instead communicates with
the jury by writing queries to standard output and reading the answers from
standard input.

You can perform up to $50$ operations to find the wallet. To do this just print
\texttt{"<direction> <steps>"} to \texttt{stdout}.

\begin{description}
    \item[\texttt{<direction>}] A single character of  $'n'$ (North), $'e'$ (East) , $'s'$ (South) or $'w'$ (West) describing the walking direction.
    \item[\texttt{<steps>}] A positive integer that sets the amount of steps Vitalik should walk.
\end{description}

In the beginning, Vitalik already pushed the button to initialize the transmitter.
Then he will walk based on your directions and press the button again.
You will get \texttt{hot}, \texttt{cold} or \texttt{found} as an answer.

\paragraph*{Sample interaction}

\noindent\begin{tcblisting}{listing only, colframe=white, colback=black!10!white, sharp corners, box align=top}
< w 20
> cold
< e 50
> hot
< s 15
> cold
< n 55
> found

\end{tcblisting}

\end{document}